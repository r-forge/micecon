\section{Sample selection}


\subsection{The problem}

Almost always, the humans can to a certain degree affect their fate.
Almoust never can we control for all the information they used for
their decisions.  Hence endogeneity problems are there in every micro
dataset.


\subsection{Possible solutions}

Because we want to break the endogeneity problem, we have to use
additional information.

\begin{itemize}
\item Randomisation: experiment, where humans \emph{cannot} affect
  their fate.  Seldom feasible for ethical or pracitcal reasons.
\item Instruments (exclusion restrictions): more-or-less
  randomisation.  Just not explicit, based on background, institutions
  etc.
\item Information about the functional form.  For instance,
  assumptions about the distribution of the error terms.
\end{itemize}


\subsection{Heckman models}

Heckman (1976) proposed a two-step solution to the fully-parametric
model.  He assumed bivariate normal error distribution.

The original article suggests using the (less efficient) two-step
solution for exploratory work and as the initial value for maximum
likelihood estimation -- two-step solution costs \$15 while the
maximum-likelihood solution costs \$700.

