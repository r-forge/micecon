\section{Sample selection}


\subsection{The problem}

Almost always, the humans can to a certain degree affect their fate.
Almoust never can we control for all the information they used for
their decisions.  Hence endogeneity problems are there in every micro
dataset.


\subsection{Possible solutions}

Because we want to break the endogeneity problem, we have to use
additional information.

\begin{itemize}
\item Randomisation: experiment, where humans \emph{cannot} affect
  their fate.  Seldom feasible for ethical or pracitcal reasons.
\item Instruments (exclusion restrictions): more-or-less
  randomisation.  Just not explicit, based on background, institutions
  etc.
\item Information about the functional form.  For instance,
  assumptions about the distribution of the error terms.
\item Timing information.  In some cases, relative timing of the
  events may give us information, necessary for identification of the
  causal effect.
\end{itemize}

During the recent decades, either randomisation or the
pseudo-randimisation (natural experiments) has become the state-of-the
art while estimating the causal effects.  The methods, relying on the
distributional assumptions are becoming less widely used.  The reason
is obvious -- the parametric assumptions can only seldom be justified,
and we don't want our results to rely on dubious assumptions.  Even if
we have valid instruments or exclusion restrictions, the parametric
hypothesis add their interference which may be sometimes hard to
detect.  

However, even if these models are not popular in research any more,
they still serve as excellent teaching tools.  Heckman-type selection
models easily allow us to experiment with selection bias,
misspecification, exclusion restrictions etc.  These models are easy
to visualise and understand.


\subsection{Heckman models}

\citet{heckman1976} proposed a two-step solution to the fully-parametric
model.  He assumed bivariate normal error distribution.

The original article suggests using the (less efficient) two-step
solution for exploratory work and as the initial value for maximum
likelihood estimation -- two-step solution costs \$15 while the
maximum-likelihood solution costs \$700.

