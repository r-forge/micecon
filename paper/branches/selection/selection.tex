\documentclass[article]{jss}

\author{Arne Henningsen\\Kiel University
  \And
  Ott Toomet\\Tartu University}
\title{Sample Selection Models in \proglang{R}:\\
  Package \pkg{micEcon}}

%% for pretty printing and a nice hypersummary also set:
\Plainauthor{Arne Henningsen, Ott Toomet} %% comma-separated
\Plaintitle{Sample Selection Models in R: Package micEcon} %% without formatting
\Shorttitle{Sample Selection Models in R} %% a short title (if necessary)

%% an abstract and keywords
\Abstract{
  Heckman-style sample selection models, and their implementation in package
  \pkg{micEcon} for \proglang{R}, are discussed.
}
\Keywords{sample-selection models, Heckman selection models, econometrics, \proglang{R}}
\Plainkeywords{sample-selection models, Heckman selection models, econometrics, R}

%% publication information
%% NOTE: This needs to filled out ONLY IF THE PAPER WAS ACCEPTED.
%% If it was not (yet) accepted, leave them commented.
%% \Volume{13}
%% \Issue{9}
%% \Month{September}
%% \Year{2004}
%% \Submitdate{2004-09-29}
%% \Acceptdate{2004-09-29}

%% The address of (at least) one author should be given
%% in the following format:
\Address{
  Ott Toomet\\
  Department of Economics\\
  Tartu University\\
  Narva 4-A123\\
  Tartu 51009, Estonia\\
  Telephone: +372 737 6348\\
  E-mail: \email{otoomet@ut.ee}\\
  URL: \url{http://www.obs.ee/~siim/}
}
%% It is also possible to add a telephone and fax number
%% before the e-mail in the following format:
%% Telephone: +43/1/31336-5053
%% Fax: +43/1/31336-734

%% for those who use Sweave please include the following line (with % symbols):
%% need no \usepackage{Sweave.sty}

%% end of declarations %%%%%%%%%%%%%%%%%%%%%%%%%%%%%%%%%%%%%%%%%%%%%%%


\begin{document}

%% include your article here, just as usual
%% Note that you should use the \pkg{}, \proglang{} and \code{} commands.


\section{Sample selection}


\subsection{The problem}

Almost always, the humans can to a certain degree affect their fate.
Almoust never can we control for all the information they used for
their decisions.  Hence endogeneity problems are there in every micro
dataset.


\subsection{Possible solutions}

Because we want to break the endogeneity problem, we have to use
additional information.

\begin{itemize}
\item Randomisation: experiment, where humans \emph{cannot} affect
  their fate.  Seldom feasible for ethical or pracitcal reasons.
\item Instruments (exclusion restrictions): more-or-less
  randomisation.  Just not explicit, based on background, institutions
  etc.
\item Information about the functional form.  For instance,
  assumptions about the distribution of the error terms.
\end{itemize}


\subsection{Heckman models}

Heckman (1976) proposed a two-step solution to the fully-parametric
model.  He assumed bivariate normal error distribution.

The original article suggests using the (less efficient) two-step
solution for exploratory work and as the initial value for maximum
likelihood estimation -- two-step solution costs \$15 while the
maximum-likelihood solution costs \$700.



\section[Implementation]{Implementation in \pkg{micEcon}}



\section[Usage]{Using the \code{selection} function}



%% Note: If there is markup in \(sub)section, then it has to be escape as above.

\end{document}