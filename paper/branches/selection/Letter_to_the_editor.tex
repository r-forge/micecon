% -*- coding: utf-8 -*-
\documentclass[a4paper]{article}
\usepackage[T1]{fontenc}
\usepackage[utf8]{inputenc}
\usepackage{amsmath}
\usepackage{graphicx}
\usepackage{natbib}
\usepackage{icomma,xspace}
% \input{isomath}
\renewcommand*{\vec}[1]{\boldsymbol{#1}}% vector

\title{Letter to the Editor}
\author{Arne Henningsen \and Ott Toomet}

\begin{document}

\maketitle

Dear Roger Koenker,

\bigskip

Hereby we send the revised version of our paper ``Sample Selection
Models in R: Package micEcon''.  We appreciate your and the referees'
comments very much.  Below, we briefly describe the changes from
the previous version, according to your and the referees' suggestions.

\paragraph{Your comments}

\begin{itemize}
\item[1]
  \begin{itemize}
  \item[a)] You are right.  We have called these ``simulations''.
  \item[b)] We have reduced the output, presented in the paper.  In
    several cases, we just report the estimated coefficients, not the
    full summaries.
  \item[c)] We have re-worded the final part of section 4.1.
  \item[d)] We are afraid, that introducing (real) Monte-Carlo
    experiments will shift the focus of the paper away from our
    implementation of selection models.  We don't think we can
    contribute to the literature here because, as you also pointed,
    the properties of the estimator are already well-known.    
  \end{itemize}
\item[2] We have added a short discussion of more general selection
  models and necessary API generalisations.
\item[3] Our paper includes two examples with non-Gaussian
  disturbances (the last examples in Section 4.2).  Although more
  types of misspecification may be added easily, we are not sure
  whether it helps to improve the paper.  The paper already includes a
  fair number of various examples, adding even more may not improve
  the readers' understanding of the package's capabilities while
  possibly adding to the complexity of presentation.
\item[4] The Section 5 is retitled to ``Two Replication Excercises''.
\item[5] The Section 6 is retitled to ``Robustness Issues''.  We also
  introduced a way to switch maximisation algorithms, and discuss the
  non-convergence example (Section 6.1) in more detail.  We describe a
  usable workaround, involving more robust maximisation methods.
\end{itemize}

\paragraph{First referee's comments}

\begin{itemize}
\item[2] The reference to our package as an ``excellent tool'' is removed.
\item[3] We have re-worded the problematic sentences
\item[4] Section 5 is retitled, see above.
\end{itemize}

\paragraph{Second referee's comments}

\begin{itemize}
\item[1] Reference to the non-parametric methods as ``superior'' is
  removed.
\item[2] We have given Heckman's method it's proper credits.
\item[3] Corrected.
\item[4] We have removed the claim that the method is not popular any more.
\item[5] Corrected.
\item[6] We have re-worded the footnote about identifycation at
  infinity more carefully.  We agree that ``identification'' is a
  completely different concept than estimation on a finite sample.
  However, we still argue that in this case they ``are related''.
  \cite[p. 205]{chamberlain1986} shows, that under suitable
  assumptions the parameter is identified, ``if the distribution of
  $X$ has unbounded support''.  Hence the normality of
  $(\varepsilon^{S}, \varepsilon^{O})'$ is of less importance as large
  variability in ${\vec{\beta}^{S}}' \vec{x}^{S}$ carries additional
  information.
\item[7] Corrected
\item[8] Corrected
\item[9] The sentence is removed
\end{itemize}

\bigskip

Sincerely,

\bigskip

Arne Henningsen

Ott Toomet

\bibliographystyle{econometrica}
\bibliography{selection}

\end{document}

TODO:

R1

1) Proofreading

3) Re-write the introduction

5) add the LI & Racine (2007)

