\documentclass[12pt]{scrartcl}
\usepackage[utf8]{inputenc}
\usepackage[T1]{fontenc}
\usepackage{lmodern}

\usepackage{geometry}
\geometry{verbose, a4paper, tmargin=25mm, bmargin=30mm, lmargin=25mm,
   rmargin=25mm, headheight=0mm, headsep=0mm, footskip=13mm}

\usepackage{setspace}
\onehalfspacing

\usepackage{natbib}
\bibliographystyle{jss}

\newcommand{\code}[1]{\texttt{#1}}
\newcommand{\pkg}[1]{\mbox{\textbf{#1}}}
\newcommand{\proglang}[1]{\mbox{\textsf{#1}}}

\author{Ott Toomet and Arne Henningsen\\
  \normalsize Tartu University and University of Kiel}
\title{Sample Selection Models in \proglang{R}:\\
  Package \pkg{sampleSelection}}

\begin{document}
\maketitle

Social scientists are often interested in causal effects --- what is
the impact of a new drug, a certain type of school or being born as a
twin.  Many of these cases are not under the researcher's control.  
Often, the subjects can decide themselves, whether they take a
drug or which school they attend.  They cannot control whether they
are twins, but neither can the researcher --- the twins may tend to be
born in different types of families than singles.  All these cases
are similar from the statistical point of view.  Whatever is the
sampling mechanism, from an initial ``random'' sample we extract a
sample of interest, which may not be representative of the population
as a whole
\citep[see][p.~1937, for a discussion]{heckman+macurdy1986}.

This problem --- people who are ``treated'' may be different than the rest
of the population --- is usually referred to as a \emph{sample
  selection} or \emph{self-selection} problem.  We cannot estimate the
causal effect, unless we solve the selection
problem.
Otherwise, we will
never know which part of the observable outcome is related to the
causal relationship and which part is due to the fact that different
people were selected for the treatment and control groups.

The most popular solutions for sample selection problems
are based on \citet{heckman1976}.
A variety of generalisations of Heckman's standard sample selection model
can be found in the literature.
These models are also called ``generalised Tobit models'',
whereas Heckman's standard sample selection model
is called ``Tobit-2'' model \citep{amemiya84,amemiya1985}.
These models are popular in estimating impacts
of various factors in economics and other social sciences.
Although modern econometrics has non- and semiparametric estimation
methods in its toolbox, Heckman models are an integral part of
modern applied analysis and econometrics syllabus.

We want to describe the estimation of Heckman-type sample
selection models in \proglang{R}
and the implementation of these models
in the package \pkg{sampleSelection}.
We illustrate the usage of the package on several simulation
and real data examples.
We argue that the package can be used both in applied research and
teaching.

\bibliography{selection}

\end{document}
