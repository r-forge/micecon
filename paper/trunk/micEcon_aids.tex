\subsection{Almost Ideal Demand System}

The �Almost Ideal Demand System� (AIDS) \citep{deaton80a}
is the most popular demand system in empirical demand analysis.
It is based on the following expenditure function:
\begin{equation}
\ln e( p , U )
= \alpha_0
+ \sum_{i=1}^n \alpha_i \ln p_i
+ \frac{1}{2} \sum_{i=1}^n \sum_{j=1}^n \gamma_{ij}^* \ln p_i \ln p_j
+ \beta_0 U \prod_{i=1}^n p_i^{\beta_i}
\end{equation}
where
$e( p , U )$ is the minimum expenditure to obtain
utility level $U$
given consumer prices $p$,
$n$ is the number of goods,
$p = (p_1 , \ldots , p_n )$ are consumer prices,
and all $\alpha$s, $\beta$s und $\gamma$s are parameters.
Based on this expenditure function,
following Marshallian demand functions can be derived:
\begin{align}
x_i ( p, m )
& = \frac{m}{p_i}
   \left( \alpha_i + \sum_{j=1}^n \gamma_{ij} \ln p_j
   + \beta_i \ln \left( \frac{m}{\mathcal{P}} \right) \right)
\label{eq:aidsDemand}\\
\text{with } \ln \mathcal{P}
& = \sum_{i=1}^n \alpha_i \ln p_i
+ \frac{1}{2} \sum_{i=1}^n \sum_{j=1}^n \gamma_{ij} \ln p_i \ln p_j
\label{eq:aidsTranslog}
\end{align}
where $x_i$ is the demand for the $i$th good,
$m = \sum_{i=1}^n p_i x_i$ is total expenditure,
and $\gamma_{ij} = ( \gamma_{ij}^* + \gamma_{ji}^* ) / 2$.
These demand functions can be simplified
by multiplying both sides by $p_i / m$.
Now, we have expenditure shares $s_i = p_i x_i / m$ on the left hand side:
\begin{equation}
s_i ( p, m ) = \alpha_i + \sum_{j=1}^n \gamma_{ij} \ln p_j
   + \beta_i \ln \left( \frac{m}{\mathcal{P}} \right)
   \label{eq:aidsShares}
\end{equation}
The �adding up� restriction,
i.e.\ the expenditure shares have to sum up to one,
requires
\begin{align}
& \sum_{i=1}^n \alpha_i = 1\\
& \sum_{i=1}^n \beta_i = 0\\
& \sum_{i=1}^n \gamma_{ij} = 0 \; \forall \, j
\end{align}
Homogeneity of degree zero in prices requires
\begin{equation}
\sum_{j=1}^n \gamma_{ij} = 0 \; \forall \, i
\end{equation}
And symmetry requires
\begin{equation}
\gamma_{ij} = \gamma_{ji} \; \forall \, i,j
\end{equation}

The demand functions~(\ref{eq:aidsShares}) are non-linear in parameters
due to the translog price index~(\ref{eq:aidsTranslog}).
However, if the translog price index~(\ref{eq:aidsTranslog})
is approximated by a price index
that does not depend on any estimated parameters,
the demand functions become linear in parameters.
Althoug this �Linear Approximation� (LA-AIDS) simplifies econometric estimation,
it introduces several other theoretical and econometric problems.

The package \pkg{micEcon} provides many functions for demand analysis
with the AIDS.
For instance, it can be used for:
\begin{itemize}
\item econometric estimation using
   the popular but inaccurate �Linear Approximation� (LA-AIDS)
   and the accurate �Iterated Linear Least Squares Estimator� (ILLE)
   \citep{blundell99}
\item searching for the intercept of the translog price index ($\alpha_0$)
   that gives the best fit to the model
\item optional imposition of homogeneity and symmetry
\item checking for monotonicity and concavity
\item calculating predicted quantities and expenditure shares
\item calculating Marshallian (uncompensated) price elasticities,
   Hicksian (compensated) price elasticities,
   income (expenditure) elasticities, and their standard errors
\end{itemize}
