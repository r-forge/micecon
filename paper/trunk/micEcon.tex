\documentclass[12pt,english]{article}
\usepackage{pslatex}
\usepackage[T1]{fontenc}
\usepackage[latin1]{inputenc}
\usepackage{geometry}
\geometry{verbose,a4paper,tmargin=2cm,bmargin=2cm,lmargin=2.5cm,rmargin=2.5cm,headheight=0cm,headsep=0cm,footskip=0cm}
\pagestyle{empty}

\usepackage{amsmath}

\usepackage{Sweave}

\usepackage{setspace}
\onehalfspacing

\usepackage[authoryear]{natbib}

\usepackage{csquotes}
\MakeOuterQuote{�}

\makeatletter
\usepackage{url}

\usepackage{babel}

\newcommand{\pkg}[1]{{\normalfont\fontseries{b}\selectfont #1}}
\let\proglang=\textsf
\newcommand{\code}{\texttt}

\clubpenalty=10000
\widowpenalty=10000

\makeatother



\makeatother

\begin{document}
\begin{center}
\textbf{\LARGE
Microeconomic Analysis with \proglang{R}
}

\bigskip
{\Large
Arne Henningsen and Ott Toomet
}
\end{center}

Since its first public release in 1993, the free open source statistical
language and development environment �\proglang{R}� \citep{r-project}
has been increasingly used for statistical analysis.
While it has been prelevant in many scientific disciplines for a long time,
it was not very widespread in economics in the first years.
However, this situation has changed in recent years.
Consequently, the number of extension packages for \proglang{R}
that are suitable for economists
has strongly increased in the last few years.
One of these packages is called �\pkg{micEcon}� \citep{r-micecon} and provides
tools for microeconomic analysis.

Initially, the \pkg{micEcon} package included only tools
for microeconomic modeling.
For example, it provides functions for demand analysis with the
�Almost Ideal Demand System� (AIDS) \citep{deaton80a}.
These functions enable the econometric estimation,
calculation of demand (price and income/expenditure) elasticities and
checks for theoretical consistency
by one single \proglang{R} command.
Second, \pkg{micEcon} contains tools for production analysis with the
�Symmetric Normalized Quadratic� (SNQ) profit function
\citep{diewert87,diewert92,kohli93}.  
Additionally to the econometric estimation and calculation of price elasticities,
it includes a function that imposes convexity on the
estimated profit function using a new sophisticated method proposed by
\citet{koebel03}.
Third, this package provides a convenient interface to �FRONTIER 4.1�, Tim
Coelli's software for stochastic frontier analysis \citep{coelli96}.
Furthermore, \pkg{micEcon} includes tools for other functional forms, namely 
translog and quadratic functions.

About a year ago, we have added tools for sample selection models
that are also often applied in microeconomic analyses.
The \pkg{micEcon} package now includes functions to estimate these models
using the two-step Heckman or an efficient maximum likelihood procedure.
Furthermore, tools to calculate selectivity terms (�inverse Mill's ratios�)
even from bivariate probit models have been added.

On the useR! conference,
we would like to present the capabilities of the \pkg{micEcon} package.
Applied economists interested in microeconomic modeling will be invited to
join our team and contribute to this package
by providing tools for other types of microeconomic analyses.


\bibliographystyle{jss}
%\bibliography{/home/suapm095/Documents/Literatur/arne}
\bibliography{micEcon}

\end{document}
