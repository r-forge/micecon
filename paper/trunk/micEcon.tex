\documentclass[12pt,english]{article}
\usepackage{pslatex}
\usepackage[T1]{fontenc}
\usepackage[latin1]{inputenc}
\usepackage{geometry}
\geometry{verbose,a4paper,tmargin=2cm,bmargin=2cm,lmargin=2.5cm,rmargin=2.5cm,headheight=0cm,headsep=0cm,footskip=0cm}
\pagestyle{empty}

\usepackage{amsmath}

\usepackage{Sweave}

\usepackage{setspace}
\onehalfspacing

\usepackage[authoryear]{natbib}

\usepackage{csquotes}
\MakeOuterQuote{�}

\makeatletter
\usepackage{url}

\usepackage{babel}

\newcommand{\pkg}[1]{{\normalfont\fontseries{b}\selectfont #1}}
\let\proglang=\textsf
\newcommand{\code}{\texttt}

\clubpenalty=10000
\widowpenalty=10000

\makeatother



\makeatother

\begin{document}
\begin{center}
\textbf{\LARGE
Microeconomic Analysis with \proglang{R}
}
\vspace{3em}\\
\parbox[t]{0.49 \textwidth}{\centering
\textbf{Arne Henningsen}\\
Department of Agricultural Economics\\
University of Kiel\\
Olshausenstr. 40\\
D-24098 Kiel (Germany)\\
Telephone: +49-431-880 4445\\
E-mail: \url{ahenningsen@email.uni-kiel.de}\\
URL: \url{http://www.uni-kiel.de/agrarpol/ahenningsen/}
}
\parbox[t]{0.49 \textwidth}{\centering
\textbf{Ott Toomet}\\
Department of Economics\\
Tartu University\\
Narva 4-A123\\
Tartu 51009, Estonia\\
Telephone: +372 737 6348\\
E-mail: \url{otoomet@ut.ee}\\
URL: \url{http://www.obs.ee/~siim/}
}
\vspace{3em}\\
\end{center}

\section{Introduction}
Since its first public release in 1993, the free open source statistical
language and development environment �\proglang{R}� \citep{r-project}
has been increasingly used for statistical analysis.
While it has been prevalent in many scientific disciplines for a long time,
it was not very widespread in economics in the first years.
However, this situation has changed in recent years.
Consequently, the number of extension packages for \proglang{R}
that are suitable for economists
has strongly increased in the last few years.
One of these packages is called �\pkg{micEcon}� \citep{r-micecon} and provides
tools for microeconomic analysis.


\section{Microeconomic Modeling}

The \pkg{micEcon} package includes several tools for microeconomic modeling.
For instance, these models include
the �Almost Ideal Demand System� (AIDS) \citep{deaton80a},
the �Symmetric Normalized Quadratic� (SNQ) profit function
\citep{diewert87,diewert92,kohli93},
and stochastic frontier models \citep{coelli96}.


\subsection{Almost Ideal Demand System}

The �Almost Ideal Demand System� (AIDS) \citep{deaton80a}
is the most popular demand system in empirical demand analysis.
\pkg{micEcon} provides functions for:
\begin{itemize}
\item econometric estimation using
   the popular but inaccurate �Linear Approximation� (LA-AIDS)
   and the accurate �Iterated Linear Least Squares Estimator� (ILLE)
   \citep{blundell99}
\item searching for the intercept of the translog price index ($\alpha_0$)
   that gives the best fit to the model
\item optional imposition homogeneity and symmetry
\item checking for monotonicity and concavity
\item calculating predicted quantities and expenditure shares
\item calculating Marshallian (uncompensated) price elasticities,
   Hicksian (compensated) price elasticities,
   income (expenditure) elasticities, and their standard errors
\end{itemize}


\subsection{Symmetric Normalized Quadratic Profit Function}

The �Symmetric Normalized Quadratic� (SNQ) profit function
\citep{diewert87,diewert92,kohli93}
is an advanced model for production analysis.
It is often used in microeconomic production models
because it allows the imposition of global convexity.
\pkg{micEcon} provides functions for:
\begin{itemize}
\item econometric estimation
\item imposition of global convexity using a recent two-step
   estimation procedure \citep{koebel03}
\item calculating price elasticities of inputs and outputs and their standard errors
\item calculating fixed factor elasticities
\item calculating shadow prices of fixed factors
\item calculating Hessian matrices and their derivatives with respect
   to coefficients
\item calculating predicted profits and netput quantities including
   standard errors and confidence limits of prediction
\end{itemize}


\subsection{Stochastic Frontier Models}

Stochastic frontier analysis is a popular tool for production analysis.
It can simultaneously model production technology
as well as possible technical (and allocative) inefficiencies.
\pkg{micEcon} provides a convenient interface to
Tim Coelli's software for stochastic frontier analysis
�FRONTIER 4.1� \citep{coelli96}.
It can be used to write input files for and read output files of �FRONTIER 4.1�.
Hence, the model or the data can be easily modified in \proglang{R}
and the estimation results can be conveniently used
for further analysis.


\subsection{Other Microeconomic Model}

Furthermore, \pkg{micEcon} includes tools for other functional forms,
namely translog and quadratic functions.
\pkg{micEon} provides functions for
\begin{itemize}
\item econometric estimation
\item calculating derivatives with respect to regressors
\item calculating (bordered) Hessian matrices with respect to regressors
\end{itemize}

\section{Sample Selection Models}

\section{Sample selection}


\subsection{The problem}

Almost always, the humans can to a certain degree affect their fate.
Almoust never can we control for all the information they used for
their decisions.  Hence endogeneity problems are there in every micro
dataset.


\subsection{Possible solutions}

Because we want to break the endogeneity problem, we have to use
additional information.

\begin{itemize}
\item Randomisation: experiment, where humans \emph{cannot} affect
  their fate.  Seldom feasible for ethical or pracitcal reasons.
\item Instruments (exclusion restrictions): more-or-less
  randomisation.  Just not explicit, based on background, institutions
  etc.
\item Information about the functional form.  For instance,
  assumptions about the distribution of the error terms.
\item Timing information.  In some cases, relative timing of the
  events may give us information, necessary for identification of the
  causal effect.
\end{itemize}

During the recent decades, either randomisation or the
pseudo-randimisation (natural experiments) has become the state-of-the
art while estimating the causal effects.  The methods, relying on the
distributional assumptions are becoming less widely used.  The reason
is obvious -- the parametric assumptions can only seldom be justified,
and we don't want our results to rely on dubious assumptions.  Even if
we have valid instruments or exclusion restrictions, the parametric
hypothesis add their interference which may be sometimes hard to
detect.  

However, even if these models are not popular in research any more,
they still serve as excellent teaching tools.  Heckman-type selection
models easily allow us to experiment with selection bias,
misspecification, exclusion restrictions etc.  These models are easy
to visualise and understand.


\subsection{Heckman models}

\citet{heckman1976} proposed a two-step solution to the fully-parametric
model.  He assumed bivariate normal error distribution.

The original article suggests using the (less efficient) two-step
solution for exploratory work and as the initial value for maximum
likelihood estimation -- two-step solution costs \$15 while the
maximum-likelihood solution costs \$700.


% -*- TeX-master: "selection", TeX-PDF-mode: t -*-

\section[Implementation]{Implementation in \pkg{micEcon}}

The estimation of the Heckman selection models are done by maximum
likelihood (ML).  The initial values are calculated by two-step methods,
unless the user prevides the values herself.  

The main frontend for the selection models is the command
\code{selection}.  It needs a formula for the selection equation, and
one (or a list of two for tobit-5 models) for the outcome equation.
One can choose the method to be either ``ml'' for the MLf estimation,
or ``2step'' for two-step results.  If the user does not provide the
initial values for the ML estimate (which is probably almoust always
the case), the \code{selection} calculates the consistent initial
values by either \code{heckit2} or \code{heckit5}.  These initial
values are transfer further to the ML estimation function
\code{heckit2fit} or \code{heckit5fit}.

\section[Usage]{Using the \code{selection} function}

In this subsection we show a few examples on the typical usage of
\code{selection}.  The first example uses a correctly specified model
with exclusion restriction:

\begin{Code}
  N <- 1000
  library(mvtnorm)
  vc <- diag(3)
  vc[lower.tri(vc)] <- c(0.9, 0.5, 0.6)
  vc[upper.tri(vc)] <- vc[lower.tri(vc)]
  eps <- rmvnorm(N, rep(0, 3), vc)
  xs <- runif(N)
  ys <- xs + eps[,1] > 0
  xo1 <- runif(N)
  yo1 <- xo1 + eps[,2]
  xo2 <- runif(N)
  yo2 <- xo2 + eps[,3]
  a <- selection(ys~xs, list(yo1 ~ xo1, yo2 ~ xo2), print.level=print.level)
  summary(a)
\end{Code}

Here we use the \pkg{mvtnorm} library, in order to create 3D normal
error terms.  We specify the correlation between the selection error
term and the outcome errors as $0.9$ and $0.6$ respectively.  Further,
we generate the explanatory variable \code{xs} for the selection
equation.  Here we have the exclusion restriction as the \code{xs} is
distinct from the explanatory variables in the outcome equation,
\code{xo1} and \code{xo2}.  All the three regressions are done in the
simplest possible for with only one explanatory variable where we add
a normal error term with variance equal to unity.  In the case of
selection, we have to transform the continuous outcome to a binary
one.

The result should look something like that:
\begin{Code}
   --------------------------------------------
   Maximum Likelihood estimation
   Newton-Raphson maximisation, 2 iterations
   Return code 3: Last step could not find a value above the current.
   May be near a solution
   Log-Likelihood: -1892.149 
   10  free parameters
   Estimates:
   coef        stdd           t    P(|b| > t)
   (Intercept)    0.13296701 0.056933090   2.3354961  1.951752e-02
   xs             0.85615385 0.083559336  10.2460585  1.232681e-24
   X1(Intercept)  0.02640296 0.059200948   0.4459888  6.556053e-01
   X1xo1          0.95331865 0.080630715  11.8232692  2.959396e-32
   sigma1         0.91644892 0.029150613  31.4384098 6.045596e-217
   rho1           0.96879226 0.003442300 281.4375216  0.000000e+00
   X2(Intercept) -0.11349697 0.065541444  -1.7316825  8.333011e-02
   X2xo2          1.06464221 0.104674935  10.1709375  2.673233e-24
   sigma2         1.01090379 0.027895950  36.2383712 1.515094e-287
   rho2           0.84162026 0.021087135  39.9115503  0.000000e+00
   --------------------------------------------
\end{Code}
One can see, that the main parameters of interest -- \code{X1xo1} and
\code{X2xo2} are quite precisely estimated.  However, in the current
example, the estimate for probit intercept, and the parameters for the
error distributions, are biased.

The \pkg{micEcon} package now includes functions to estimate these models
using the two-step Heckman or an efficient maximum likelihood procedure.
Furthermore, tools to calculate selectivity terms (�inverse Mill's ratios�)
even from bivariate probit models have been added.

\section{Conclusion}
On the useR! conference,
we would like to present the capabilities of the \pkg{micEcon} package.
Applied economists interested in microeconomic modeling will be invited to
join our team and contribute to this package
by providing tools for other types of microeconomic analyses.


\bibliographystyle{jss}
%\bibliography{/home/suapm095/Documents/Literatur/literatur}
\bibliography{micEcon}

\end{document}
