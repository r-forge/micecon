\documentclass[12pt,english]{article}
\usepackage{pslatex}
\usepackage[T1]{fontenc}
\usepackage[latin1]{inputenc}
\usepackage{geometry}
\geometry{verbose,a4paper,tmargin=2cm,bmargin=2cm,lmargin=2.5cm,rmargin=2.5cm,headheight=0cm,headsep=0cm,footskip=0cm}
\pagestyle{empty}

\usepackage{amsmath}

\usepackage{Sweave}

\usepackage{setspace}
\onehalfspacing

\usepackage[authoryear]{natbib}

\usepackage{csquotes}
\MakeOuterQuote{�}

\makeatletter
\usepackage{url}

\usepackage{babel}

\newcommand{\pkg}[1]{{\normalfont\fontseries{b}\selectfont #1}}
\let\proglang=\textsf
\newcommand{\code}{\texttt}

\clubpenalty=10000
\widowpenalty=10000

\makeatother



\makeatother

\begin{document}
\begin{center}
\textbf{\LARGE
Microeconomic Analysis with \proglang{R}
}

\bigskip
{\Large
Arne Henningsen and Ott Toomet
}
\end{center}

\section{Introduction}
Since its first public release in 1993, the free open source statistical
language and development environment �\proglang{R}� \citep{r-project}
has been increasingly used for statistical analysis.
While it has been prelevant in many scientific disciplines for a long time,
it was not very widespread in economics in the first years.
However, this situation has changed in recent years.
Consequently, the number of extension packages for \proglang{R}
that are suitable for economists
has strongly increased in the last few years.
One of these packages is called �\pkg{micEcon}� \citep{r-micecon} and provides
tools for microeconomic analysis.


\section{Microeconomic Modeling}

The \pkg{micEcon} package includes several tools for microeconomic modeling.
For instance, these models include
the �Almost Ideal Demand System� (AIDS) \citep{deaton80a},
the �Symmetric Normalized Quadratic� (SNQ) profit function
\citep{diewert87,diewert92,kohli93},
and stochastic frontier models \citep{coelli96}.


\subsection{Almost Ideal Demand System}

The �Almost Ideal Demand System� (AIDS) \citep{deaton80a}
is the most popular demand system in empirical demand analysis.
\pkg{micEcon} provides functions for:
\begin{itemize}
\item econometric estimation using
   the popular but inaccurate �Linear Approximation� (LA-AIDS)
   and the accurate �Iterated Linear Least Squares Estimator� (ILLE)
   \citep{blundell99}
\item searching for the intercept of the translog price index ($alpha_0$)
   that gives the best fit to the model
\item optional imposition homogeneity and symmetry
\item checking for monotonicity and concavity
\item predicted quantities and expenditure shares
\item Marshallian (uncompensated) price elasticities,
   Hicksian (compensated) price elasticities,
   income (expenditure) elasticities, and their standard errors
\end{itemize}


\subsection{Symmetric Normalized Quadratic Profit Function}

The �Symmetric Normalized Quadratic� (SNQ) profit function
\citep{diewert87,diewert92,kohli93}
is an advanced model for production analysis.
It is often used in microeconomic production models
because it allows the imposition of global convexity.
\pkg{micEcon} provides functions for:
\begin{itemize}
\item econometric estimation
\item imposition of global convexity using a new  sophisticated two-step
   estimation procedure \citep{koebel03}
\item price elasticities of inputs and outputs and their standard errors
\item fixed factor elasticities
\item shadow prices of fixed factors
\item Hessian matrix and its derivatives with respect to coefficients
\item predicted profits and netput quantities including the standard errors
   and the confidence limits of prediction
\end{itemize}


\subsection{Stochastic Frontier Models}

Stochastic frontier analysis is a popular tool for production analysis.
It can simultaneously model production technology
as well as possible technical (or allocative) inefficiencies.
\pkg{micEcon} provides a convenient interface to �FRONTIER 4.1�,
Tim Coelli's software for stochastic frontier analysis \citep{coelli96}.
It can be used to write input files and read output files of �FRONTIER 4.1�.
Hence, the model or the data can be easily modified in \proglang{R}
and the estimation results can be conveniently used
for further analysis.


\subsection{Other Microeconomic Model}

Furthermore, \pkg{micEcon} includes tools for other functional forms,
namely translog and quadratic functions.


\section{Sample Selection Models}

About a year ago, we have added tools for sample selection models
that are also often applied in microeconomic analyses.
The \pkg{micEcon} package now includes functions to estimate these models
using the two-step Heckman or an efficient maximum likelihood procedure.
Furthermore, tools to calculate selectivity terms (�inverse Mill's ratios�)
even from bivariate probit models have been added.

\section{Conclusion}
On the useR! conference,
we would like to present the capabilities of the \pkg{micEcon} package.
Applied economists interested in microeconomic modeling will be invited to
join our team and contribute to this package
by providing tools for other types of microeconomic analyses.


\bibliographystyle{jss}
%\bibliography{/home/suapm095/Documents/Literatur/literatur}
\bibliography{micEcon}

\end{document}
