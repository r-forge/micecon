%\documentclass{beamer}
\documentclass[notes=show,handout]{beamer}
%\documentclass[handout]{beamer}

\usetheme[width=1.5cm]{PaloAlto}
% \usecolortheme{dolphin}
\logo{\pgfimage[width=1.5cm]{useR}}
\usepackage[english]{babel}
\usepackage[latin1]{inputenc}
%\usepackage[T1]{fontenc}
\usepackage{fancyvrb}
% \usepackage{beamerthemesplit}
\usepackage{csquotes}
\MakeOuterQuote{�}

\makeatletter
\setbeamertemplate{note page}[plain]
\setbeamerfont{note page}{size=\footnotesize}
\hyphenpenalty = 5000
\binoppenalty = 10000
\relpenalty = 5000
\newcommand{\myhrule}{\vspace{1ex} \hrule \vspace{1ex}}
%\setbeamersize{text margin left=0.5cm}
\makeatother




\title[micEcon]{
   \textbf{micEcon}\\ Microeconomic Analysis with R}
\author{Arne Henningsen (University of Kiel, Germany)\\[0.4em]
   Ott Toomet (University of Aarhus, Denmark)}
\date{Vienna, June 16, 2006}

\begin{document}
\frame{ \titlepage }
\note{
\begin{itemize}
\item Now I will shortly present the R package micEcon.
\item This package provides tools for microeconomic analysis.
\item It is written by me and by my co-author Ott Toomet.
\end{itemize}
}

% ============== Introduction ==============================
% \section{Introduction}
% 
% \frame{
%    \frametitle{Introduction}
% \texttt{R} for Economists
% \begin{itemize}
% \setlength{\itemsep}{0.5em}
% \item first years: \texttt{R} was not very widespread in economics
% \item this situation has changed in recent years,\\
% \item many new packages for economists
% \item \texttt{micEcon}: tools for microeconomic analysis
% \end{itemize}
% }
% \note{
% \begin{itemize}
% \item \texttt{R} has been prelevant in many scientific disciplines
%    for a long time.
% \item However, in the first years it was not very widespread in economics
% \item but this situation has changed in recent years.
% \item For instance, the number of extension packages on CRAN
%    that are suitable for economists and econometricians
%    has strongly increased in the last few years.
% \item One of these packages is called �\texttt{micEcon}�
% \item and it provides -- as I have already said -- tools for microeconomic 
%    analysis.
% \end{itemize}
% }

% ----------------- Tools ----------------------------
\section{Tools}
\subsection{Main}
\frame{
   \frametitle{Main Tools}
\texttt{micEcon} provides mainly tools for
\begin{itemize}
\setlength{\itemsep}{0.5em}
\item demand analysis with the �Almost Ideal Demand System�~(AIDS)
\item production analysis with the �Symmetric Normalized Quadratic�~(SNQ)
   profit function
\item Heckman selection models
\end{itemize}
}
\note{
\begin{itemize}
\item This package provides tools mainly for three types of analyses.
\item First, it includes several functions for econometric demand analysis
   with the �Almost Ideal Demand System�, AIDS, of Deaton and Muellbauer.
\item The AIDS has many desirable properties and is, therefore,
   the most popular demand system in empirical demand analysis.
\item Second, micEcon provides functions for production analysis
   with the �Symmetric Normalized Quadratic� profit function
   of Diewert and Wales.
% \item Formerly, this functional form was also traded under the name
%    �symmetric generalized McFadden� function.
\item This profit function is often used in microeconomic production models
   because it allows the imposition of global convexity,
   which is required by microeconomic theory.
\item Third, micEcon provides functions for Heckman selection models.
\item These selection models are required to obtain unbiased estimates,
   if a non-random subsample is used for econometric estimation.
\item These models can be estimated
   either with the traditional two-step procedure of Heckman
   or with an efficient Maximum Likelihood procedure.
\end{itemize}
}

\subsection{Miscellaneous}
\frame{
   \frametitle{Miscellaneous Tools}
\texttt{micEcon} provides some further tools for
\begin{itemize}
\item quadratic functions
\item translog functions
\item FRONTIER~4.1 (software for stochastic frontier analysis)
\item Optimization using the Newton-Raphson and BHHH algorithm
\item creating and modifying matrices
\end{itemize}
}
\note{
\begin{itemize}
\item Additionally, micEcon includes several miscellaneous functions
   that also might be useful in microeconomic analyses.
\item For instance, micEcon can be used to easily estimate
   quadratic and translog functions,
   and to do some calculations on the results
   that are relevant for economists.
%    and to calculate their derivatives with respect to the arguments.
% \item In economics, these derivatives are for instance marginal products
%    or marginal costs, 
%    which play an important role in many microeconomic analyses.
\item Moreover, micEcon provides functions to conveniently write input files
   and read output files of FRONTIER~4.1 -- a widely used software 
   for stochastic frontier analysis.
\item Optimization using the Newton-Raphson and the Berndt-Hall-Hall-and-Hausman
   algorithm can also be performed with micEcon.
\item Finally, micEcon includes some functions to create and modify matrices.
\end{itemize}
}

% ============== Demand Analysis ==============================
\section{Demand Analysis}
\frame{
   \frametitle{�Almost Ideal Demand System�}
\texttt{micEcon} provides functions for
\begin{itemize}
\item econometric estimation using the
\begin{itemize}
\item popular but inaccurate �Linear Approximation� (LA-AIDS)
\item accurate �Iterated Linear Least Squares Estimator� (ILLE)
\end{itemize}
\item optional imposition homogeneity and/or symmetry
\item checking for monotonicity and concavity
\item price elasticities, income elasticities,
   and their standard errors
\item \ldots
\end{itemize}
}
\note{
\begin{itemize}
\item Since the time constraint doesn't allow 
   that I explain all features of micEcon in-depth,
\item I focus on my favorite tool in micEcon:
   demand analysis with the �Almost Ideal Demand System�.
\item Most empirical studies estimate the non-linear demand equations of
   the AIDS by applying a linear approximation.
\item However, some studies have showed that this approximation
   leads to rather inaccurate results.
\item Therefore, Blundell and Robin proposed an �Iterated Linear Least
   Squares Estimator� that gives very accurate results.
\item micEcon can apply both estimation methods.
\item Moreover, the user can specify if the homogeneity and symmetry conditions
   that are derived from microeconomic theory should be imposed.
\item After estimation, it can be checked to what extent the monotonicity and
   concavity conditions that are also derived from microeconomic theory
   are fulfilled.
\item Finally, micEcon calculates income elasticities,
   Marshallian (uncompensated) price elasticities,
   Hicksian (compensated) price elasticities,
   and their standard errors.
\item There are some more features available
   that I won't describe here.
\end{itemize}
}


% ----------------- Demand Analysis: Commands ----------------------------
\subsection{Commands}
\begin{SaveVerbatim}{specNames}
prices <- c("pMeat", "pFruit", "pCereal", "pMisc")
shares <- c("wMeat", "wFruit", "wCereal", "wMisc")
\end{SaveVerbatim}
% \end{Verbatim}
\begin{SaveVerbatim}{estSystem}
estResult <- aidsEst( prices, shares, "xFood",
   data = Blanciforti86 )
\end{SaveVerbatim}
% \end{Verbatim}
\begin{SaveVerbatim}{printResults}
print( estResult )
\end{SaveVerbatim}
% \end{Verbatim}
\frame{
   \frametitle{Demand Analysis: Commands}
\begin{itemize}
\setlength{\itemsep}{0.9em}
\item specification of the variable names:\\[0.3em]
   {\small \BUseVerbatim{specNames}}
\item estimation of the �Almost Ideal Demand System�:\\[0.3em]
   {\small \BUseVerbatim{estSystem}}
\item printing estimation results:\\[0.3em]
   {\small \BUseVerbatim{printResults}}
\end{itemize}
}
\note{
\begin{itemize}
\item Now, I will give an example of a very simple demand analysis.
\item This example comes from a paper from Blanciforti, Green and King
   that analyzes the food demand in the U.S.
\item First, the names of the variables that contain the prices and the
   expenditure shares are specified.
\item There are four food categories: meat, fruit and vegetables,
   cereal and bakery products, as well as miscellaneous food.
\item Then, the demand system is estimated.
\item The arguments are the names of the prices, the names of the expenditure
   shares, the variable name of total expenditure, and the data set.
\item Finally, the estimation results are printed.
\end{itemize}
\myhrule
}

% ----------------- Demand Analysis: Output ----------------------------
\subsection{Coefficients}
\begin{SaveVerbatim}{coefficients}
Estimation Method: Linear Approximation (LA)
with Laspeyres Index (L)

Estimated Coefficients
alpha
 wMeat wFruit wCereal  wMisc
-0.248  0.123   0.273  0.852
beta
  wMeat  wFruit wCereal   wMisc
 0.3251  0.0481 -0.0816 -0.2917
gamma
           pMeat    pFruit  pCereal   pMisc
wMeat    0.09104  -0.14571 -0.00963  0.0643
wFruit  -0.14571   0.15852  0.00594 -0.0188
wCereal -0.00963   0.00594  0.01289 -0.0092
wMisc    0.06431  -0.01875 -0.00920 -0.0364
\end{SaveVerbatim}
% \end{Verbatim}
\frame{
   \frametitle{Demand Analysis: Estimated Coefficients}
{\small \BUseVerbatim{coefficients}}
}
\note{
\begin{itemize}
\item The estimated coefficients are printed first.
\item The functional form is specified with
   a vector of so-called $\alpha$ coefficients,
   a vector of $\beta$ coefficients
   and a matrix of $\gamma$ coefficients.
\item However, the interpretation of these coefficients
   is rather limited.
\end{itemize}
}


% ----------------- Demand Analysis: Output ----------------------------
\subsection{Elasticities}
\begin{SaveVerbatim}{elasticities}
Demand Elasticities (Formula of Chalfant / Goddard)
Expenditure Elasticities
 wMeat wFruit wCereal  wMisc
 2.048  1.240   0.392  0.179

Marshallian (uncompensated) Price Elasticities
         pMeat pFruit pCereal  pMisc
wMeat   -1.032 -0.679  -0.172 -0.165
wFruit  -0.802 -0.257  -0.003 -0.179
wCereal  0.117  0.166  -0.822  0.147
wMisc    0.436  0.112   0.084 -0.811

Hicksian (compensated) Price Elasticities
         pMeat pFruit pCereal  pMisc
wMeat   -0.396 -0.269   0.103  0.562
wFruit  -0.417 -0.008   0.164  0.262
wCereal  0.239  0.245  -0.770  0.287
wMisc    0.491  0.148   0.108 -0.747
\end{SaveVerbatim}
% \end{Verbatim}
\frame{
   \frametitle{Demand Analysis: Estimated Elasticities}
{\small \BUseVerbatim{elasticities}}
}
\note{
\begin{itemize}
\item Economists like so-called elasticities very much.
\item They generally indicate a percentage change in one variable
   if another variable changes by one percent.
\item Here, the expenditure elasticities and two kinds of price elasticities
   are presented.
\item For instance, if the household increases total food expenditure by 1\%,
   the demand for meat increases by more than 2\%,
   while the demand for miscellaneous food increases only by less than .2\%.
\item Furthermore, one can see from the Marshallian Price Elasticities
   that if the price of meat increases by 1\%,
   the household buys less meat and less fruit and vegetables,
   but more cereals and more miscellaneous food.
\end{itemize}
}

% ==================== The End ===========================
\section{ }
\frame{
   \frametitle{The End . . .}

\begin{center}
{\LARGE You are invited to provide\\[0.5em]
further tools for micEcon!}\\[3em]
{\LARGE \emph{Thank you for your attention!}}
\end{center}
}
\note{
\begin{itemize}
\item You are invited to provide further tools for micEcon!
\item Thank you very much for your attention!
\end{itemize}
}

% ------------ End -------------

\end{document}
