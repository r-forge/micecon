\documentclass{beamer}
%\documentclass[notes=show,handout]{beamer}
%\documentclass[handout]{beamer}

\usetheme[width=1.5cm]{PaloAlto}
% \usecolortheme{dolphin}
\logo{\pgfimage[width=1.5cm]{useR}}
\usepackage[english]{babel}
\usepackage[latin1]{inputenc}
%\usepackage[T1]{fontenc}
\usepackage{fancyvrb}
% \usepackage{beamerthemesplit}
\usepackage{csquotes}
\MakeOuterQuote{�}

\makeatletter
\setbeamertemplate{note page}[plain]
\setbeamerfont{note page}{size=\footnotesize}
\hyphenpenalty = 5000
\binoppenalty = 10000
\relpenalty = 5000
\newcommand{\myhrule}{\vspace{1ex} \hrule \vspace{1ex}}
%\setbeamersize{text margin left=0.5cm}
\makeatother




\title[micEcon]{
   \textbf{micEcon}\\ Microeconomic Analysis with R}
\author{Arne Henningsen and Ott Toomet}
\date{University of Kiel (Germany) and\\ University of Aarhus (Denmark)}

\begin{document}
\frame{ \titlepage }
\note{
\begin{itemize}
\item Welcome!
\item authors
\item package micEcon: Tools for Microeconomic Analysis with R
\end{itemize}
}

% ============== Introduction ==============================
\section{Introduction}

\frame{
   \frametitle{Introduction}
\texttt{R} for Economists
\begin{itemize}
\setlength{\itemsep}{0.5em}
\item first years: \texttt{R} was not very widespread in economics
\item this situation has changed in recent years,\\
\item many new packages for economists
\item \texttt{micEcon}: tools for microeconomic analysis
\end{itemize}
}
\note{
\begin{itemize}
\item \texttt{R} has been prelevant in many scientific disciplines
   for a long time.
\item However, it was not very widespread in economics in the first years,
\item but this situation has changed in recent years.
\item For instance, the number of extension packages for \texttt{R}
   that are suitable for economists
   has strongly increased in the last few years.
\item One of these packages is called �\pkg{micEcon}�
\item and provides tools for microeconomic analysis.
\end{itemize}
}

% ----------------- Tools ----------------------------
\section{Tools}
\subsection{Main}
\frame{
   \frametitle{Main Tools}
\texttt{micEcon} provides mainly tools for
\begin{itemize}
\setlength{\itemsep}{0.5em}
\item demand analysis with the �Almost Ideal Demand System�~(AIDS)
\item production analysis with the �Symmetric Normalized Quadratic�~(SNQ)
   profit function
\item Heckman selection models
\end{itemize}
}
\note{
\begin{itemize}
\item The AIDS is the most popular demand system in empirical demand analysis.
\item The SNQ profit function is often used in microeconomic production models
because it allows the imposition of global convexity.
\item If a non-random (sub)sample is used for econometric estimation,
Heckman selection models are required to obtain unbiased estimates.
\end{itemize}
}

\subsection{Miscellaneous}
\frame{
   \frametitle{Miscellaneous Tools}
\texttt{micEcon} provides some further tools for
\begin{itemize}
\item creating and modifying matrices
\item quadratic functions
\item translog functions
\item FRONTIER~4.1 (software for stochastic frontier analysis)
\item maximization using the Newton-Raphson and BHHH algorithm
\end{itemize}
}
\note{
\begin{itemize}
\item quadratic functions: e.g.\ estimation,
   derivatives with respect to regressors
\item translog functions: e.g.\ estimation,
   derivatives and (bordered) Hessian matrix with respect to regressors
\item writing input files and reading output files of
FRONTIER~4.1 (software for stochastic frontier analysis)
\item optimization using the Newton-Raphson and BHHH algorithm
\end{itemize}
}

% ============== Demand Analysis ==============================
\section{Demand Analysis}
\frame{
   \frametitle{�Almost Ideal Demand System�}
\texttt{micEcon} provides functions for
\begin{itemize}
\item econometric estimation using the
\begin{itemize}
\item popular but inaccurate �Linear Approximation� (LA-AIDS)
\item accurate �Iterated Linear Least Squares Estimator� (ILLE)
\end{itemize}
\item searching for the �best� intercept of the translog price
   index~($\alpha_0$)
\item optional imposition homogeneity and symmetry
\item checking for monotonicity and concavity
\item predicted quantities and expenditure shares
\item price elasticities, income elasticities,
   and their standard errors
\end{itemize}
}
\note{
\begin{itemize}
\item econometric estimation using the
\begin{itemize}
\item popular but inaccurate �Linear Approximation� (LA-AIDS)
\item accurate �Iterated Linear Least Squares Estimator� (ILLE)
\end{itemize}
\item searching for the intercept of the translog price index ($\alpha_0$)
   that gives the best fit to the model
\item optional imposition homogeneity and symmetry
\item checking for monotonicity and concavity
\item predicted quantities and expenditure shares
\item Marshallian (uncompensated) price elasticities,
   Hicksian (compensated) price elasticities,
   income elasticities,
   and their standard errors
\end{itemize}
}


% ----------------- Demand Analysis: Commands ----------------------------
\subsection{Commands}
\begin{SaveVerbatim}{specNames}
prices <- c("pFood1", "pFood2", "pFood3", "pFood4")
shares <- c("wFood1", "wFood2", "wFood3", "wFood4")
\end{SaveVerbatim}
% \end{Verbatim}
\begin{SaveVerbatim}{estSystem}
estResult <- aidsEst( prices, shares, "xFood",
   data = Blanciforti86 )
\end{SaveVerbatim}
% \end{Verbatim}
\begin{SaveVerbatim}{printResults}
print( estResult )
\end{SaveVerbatim}
% \end{Verbatim}
\frame{
   \frametitle{Demand Analysis: Commands}
\begin{itemize}
\setlength{\itemsep}{0.9em}
\item specification of the variable names:\\[0.3em]
   {\small \BUseVerbatim{specNames}}
\item estimation of the �Almost Ideal Demand System�:\\[0.3em]
   {\small \BUseVerbatim{estSystem}}
\item printing estimation results:\\[0.3em]
   {\small \BUseVerbatim{printResults}}
\end{itemize}
}
\note{
\begin{itemize}
\item ?
\end{itemize}
}

% ----------------- Demand Analysis: Output ----------------------------
\subsection{Coefficients}
\begin{SaveVerbatim}{coefficients}
Estimation Method: Linear Approximation (LA)
with Laspeyres Index (L)

Estimated Coefficients
alpha
wFood1 wFood2 wFood3 wFood4
-0.248  0.123  0.273  0.852
beta
 wFood1  wFood2  wFood3  wFood4
 0.3251  0.0481 -0.0816 -0.2917
gamma
         pFood1   pFood2   pFood3  pFood4
wFood1  0.09104 -0.14571 -0.00963  0.0643
wFood2 -0.14571  0.15852  0.00594 -0.0188
wFood3 -0.00963  0.00594  0.01289 -0.0092
wFood4  0.06431 -0.01875 -0.00920 -0.0364
\end{SaveVerbatim}
% \end{Verbatim}
\frame{
   \frametitle{Demand Analysis: Estimated Coefficients}
{\small \BUseVerbatim{coefficients}}
}
\note{
\begin{itemize}
\item ?
\end{itemize}
}


% ----------------- Demand Analysis: Output ----------------------------
\subsection{Elasticities}
\begin{SaveVerbatim}{elasticities}
Demand Elasticities (Formula of Chalfant / Goddard)
Expenditure Elasticities
wFood1 wFood2 wFood3 wFood4
 2.048  1.240  0.392  0.179

Marshallian (uncompensated) Price Elasticities
       pFood1 pFood2 pFood3 pFood4
wFood1 -1.032 -0.679 -0.172 -0.165
wFood2 -0.802 -0.257 -0.003 -0.179
wFood3  0.117  0.166 -0.822  0.147
wFood4  0.436  0.112  0.084 -0.811

Hicksian (compensated) Price Elasticities
       pFood1 pFood2 pFood3 pFood4
wFood1 -0.396 -0.269  0.103  0.562
wFood2 -0.417 -0.008  0.164  0.262
wFood3  0.239  0.245 -0.770  0.287
wFood4  0.491  0.148  0.108 -0.747
\end{SaveVerbatim}
% \end{Verbatim}
\frame{
   \frametitle{Demand Analysis: Estimated Elasticities}
{\small \BUseVerbatim{elasticities}}
}
\note{
\begin{itemize}
\item ?
\end{itemize}
}

% ==================== The End ===========================
\section{ }
\frame{
   \frametitle{The End . . .}

\begin{center}
{\LARGE \emph{Thank you for your attention!}}
\end{center}
}
\note{
   Thank you for your attention!
}

% ------------ End -------------

\end{document}
