\subsection{Symmetric Normalized Quadratic Profit Function}

The �Symmetric Normalized Quadratic� (SNQ) profit function
\citep{diewert87,diewert92,kohli93}
is an advanced model for production analysis.
This functional form,
which is also traded under the name of
�symmetric generalized McFadden function� \citep{diewert92},
is often used in microeconomic production models
because it allows the imposition of global convexity.
The profit function is defined as follows:
\begin{equation}
\pi ( p, r )
= \sum_{i = 1}^n \alpha_i p_i
   + \frac{1}{2} w^{-1} \sum_{i = 1}^n \sum_{j = 1}^n \beta_{ij} p_i p_j
   + \sum_{i = 1}^n \sum_{j = 1}^m \delta_{ij} p_i r_j
   + \frac{1}{2} w \sum_{i = 1}^m \sum_{j = 1}^m \gamma_{ij} r_i r_j
   \label{eq:snqProfit}
\end{equation}
where
$\pi ( p, r )$ is the maximum profit that can be obtained given
netput prices $p = ( p_1 , \ldots , p_n )$
and quasi-fixed inputs $r = ( r_1 , \ldots , r_m )$,
$w = \sum_{i = 1}^n \theta_i p_i$
is a factor to normalize prices,
$\theta_i = \left. \overline{p_i \left| x_i \right|} \right/$
$\sum_{j = 1}^n \overline{ p_j \left| x_j \right| }$
are predetermined weights of the individual netput prices,
$x = ( x_1 , \ldots , x_n )$ represents the netput quantities,
$\overline{ p_i \left| x_i \right| }$ denotes the average absolute
value of netput $i$,
and all $\alpha$s, $\beta$s, $\delta$s, and $\gamma$s are the parameters.
To identify all parameters, the restrictions
$\gamma_{ij} = \gamma_{ji} \; \forall \, i,j$,
$\beta_{ij} = \beta_{ji} \; \forall \, i,j$, and
$\sum_{j = 1}^n \beta_{ij} \overline{p_j} = 0 \; \forall \, i$
are generally imposed,
where $\overline{p_j}$ are the mean prices \citep[p. 54]{diewert87}.

The corresponding netput equations can be obtained using Hotelling's Lemma:
\begin{equation}
x_i ( p , r )
= \alpha_i
   + w^{-1} \sum_{j = 1}^n \beta_{ij} p_j
   - \frac{1}{2} \theta_i w^{-2} \sum_{j = 1}^n \sum_{k = 1}^n
      \beta_{jk} p_j p_k
   + \sum_{j = 1}^m \delta_{ij} r_j
   + \frac{1}{2} \theta_i \sum_{j = 1}^m \sum_{k = 1}^m \gamma_{jk} r_j r_k
   \label{eq:snqNetput}
\end{equation}

The package \pkg{micEcon} provides many tools for production analysis
with the SNQ profit function
For instance, it can be used for:
\begin{itemize}
\item econometric estimation
\item imposition of global convexity using a recent two-step
   estimation procedure \citep{koebel03}
\item calculating price elasticities of inputs and outputs and their standard errors
\item calculating fixed factor elasticities
\item calculating shadow prices of fixed factors
\item calculating Hessian matrices and their derivatives with respect
   to coefficients
\item calculating predicted profits and netput quantities including
   standard errors and confidence limits of prediction
\end{itemize}
