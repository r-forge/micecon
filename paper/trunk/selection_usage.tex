\section[Usage]{Using the \code{selection} function}

In this subsection we show a few examples on the typical usage of
\code{selection}.  The first example uses a correctly specified model
with exclusion restriction:

\begin{Code}
  N <- 1000
  library(mvtnorm)
  vc <- diag(3)
  vc[lower.tri(vc)] <- c(0.9, 0.5, 0.6)
  vc[upper.tri(vc)] <- vc[lower.tri(vc)]
  eps <- rmvnorm(N, rep(0, 3), vc)
  xs <- runif(N)
  ys <- xs + eps[,1] > 0
  xo1 <- runif(N)
  yo1 <- xo1 + eps[,2]
  xo2 <- runif(N)
  yo2 <- xo2 + eps[,3]
  a <- selection(ys~xs, list(yo1 ~ xo1, yo2 ~ xo2), print.level=print.level)
  summary(a)
\end{Code}

Here we use the \pkg{mvtnorm} library, in order to create 3D normal
error terms.  We specify the correlation between the selection error
term and the outcome errors as $0.9$ and $0.6$ respectively.  Further,
we generate the explanatory variable \code{xs} for the selection
equation.  Here we have the exclusion restriction as the \code{xs} is
distinct from the explanatory variables in the outcome equation,
\code{xo1} and \code{xo2}.  All the three regressions are done in the
simplest possible for with only one explanatory variable where we add
a normal error term with variance equal to unity.  In the case of
selection, we have to transform the continuous outcome to a binary
one.

The result should look something like that:
\begin{Code}
   --------------------------------------------
   Maximum Likelihood estimation
   Newton-Raphson maximisation, 2 iterations
   Return code 3: Last step could not find a value above the current.
   May be near a solution
   Log-Likelihood: -1892.149 
   10  free parameters
   Estimates:
   coef        stdd           t    P(|b| > t)
   (Intercept)    0.13296701 0.056933090   2.3354961  1.951752e-02
   xs             0.85615385 0.083559336  10.2460585  1.232681e-24
   X1(Intercept)  0.02640296 0.059200948   0.4459888  6.556053e-01
   X1xo1          0.95331865 0.080630715  11.8232692  2.959396e-32
   sigma1         0.91644892 0.029150613  31.4384098 6.045596e-217
   rho1           0.96879226 0.003442300 281.4375216  0.000000e+00
   X2(Intercept) -0.11349697 0.065541444  -1.7316825  8.333011e-02
   X2xo2          1.06464221 0.104674935  10.1709375  2.673233e-24
   sigma2         1.01090379 0.027895950  36.2383712 1.515094e-287
   rho2           0.84162026 0.021087135  39.9115503  0.000000e+00
   --------------------------------------------
\end{Code}
One can see, that the main parameters of interest -- \code{X1xo1} and
\code{X2xo2} are quite precisely estimated.  However, in the current
example, the estimate for probit intercept, and the parameters for the
error distributions, are biased.