\documentclass[12pt,english]{article}
\usepackage{pslatex}
\usepackage[T1]{fontenc}
\usepackage[latin1]{inputenc}
\usepackage{geometry}
\geometry{verbose,a4paper,tmargin=2cm,bmargin=2cm,lmargin=2.5cm,rmargin=2.5cm,headheight=0cm,headsep=0cm,footskip=0cm}
\pagestyle{empty}

\usepackage{amsmath}

\usepackage{Sweave}

\usepackage{setspace}
\onehalfspacing

\usepackage[authoryear]{natbib}

\usepackage{csquotes}
\MakeOuterQuote{�}

\makeatletter
\usepackage{url}

\usepackage{babel}

\newcommand{\pkg}[1]{{\normalfont\fontseries{b}\selectfont #1}}
\let\proglang=\textsf
\newcommand{\code}{\texttt}

\clubpenalty=10000
\widowpenalty=10000

\makeatother



\makeatother

\begin{document}
\begin{center}\textbf{\LARGE Microeconomic Analysis with R}\end{center}{\LARGE \par}

\bigskip
% \begin{center}Arne Henningsen\\
% Department of Agricultural Economics\\
% University of Kiel, Germany\end{center}

Since its first public release in 1993, the free open source statistical
language and development environment ``R'' \citep{r-project} has been used more
and more for statistical analyses. 
While it is already prelevant in many scientific disciplines, 
it is not yet wide used in economics.
However, this situation has been changing in recent years. 
One cornerstone was the article ``R: Yet another econometric programming
environment'' by \citet{cribari99}.
Three years later \citet{racine02} published the article ``Using R to teach
econometrics''. 
And \citet{arai02} has a ``A brief guide to R for beginners in econometrics''
in the web that has been continuously updated and improved since that time.

The number of extension packages for special purposes has also considerably 
increased in recent years.
Merely the official website for R packages (\url{http://cran.r-project.org})
provides more than 450 different packages.
The packages cover very different areas and many of these packages are also 
useful for economists.

Over the last years the number of R packages that are useful for economists
also strongly increased. 
One of these packages is called ``micEcon'' \citep{r-micecon} and provides
tools mainly for microeconomic analysis and modeling.
For example, it includes functions for demand analyses with the 
``Almost Ideal Demand System'' (AIDS) \citep{deaton80a}.
These functions enable the econometric estimation, calculation of price and
income (expenditure) elasticities and checks for theoretical consistency
by a single R command.
Second, micEcon contains tools for production analyses with the 
``symmetric normalized quadratic'' (SNQ) profit function
\citep{diewert87,diewert92,kohli93}.  
Additionally to econometric estimation and calculation of price elasticities
and shadow prices, it includes a function that imposes convexity on the 
estimated profit function using a new sophisticated method proposed by
\citet{koebel03}.
Third, this package provides a convenient interface to ``FRONTIER 4.1'', Tim
Coelli's software for stochastic frontier analysis \citep{coelli96}.
Furthermore, micEcon includes tools for other functional forms, namely
translog and quadratic functions, and some other tools that are useful
for economists.

Free open source software like R and its packages has many advantages 
for scientists compared to commercial software.
The most important advantages are that
(a) the algorithms used by the software are publicly available and, thus,
can be checked and (peer) reviewed,
(b) these algorithms can be altered or extended to meet own requirements, and
(c) this software can be given to colleagues and students free of charge and
without any legal or other restrictions.

Hence, R is a very suitable platform for applied economists who perform
statistical and econometric analysis or develop new algorithms.

\bibliographystyle{apalike}
%\bibliography{/home/suapm095/Documents/Literatur/arne}
\bibliography{micEcon}

\end{document}
