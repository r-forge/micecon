\documentclass[12pt,english]{article}
\usepackage{pslatex}
\usepackage[T1]{fontenc}
\usepackage{geometry}
\geometry{verbose,a4paper,tmargin=2cm,bmargin=2cm,lmargin=2.5cm,rmargin=2.5cm,headheight=0cm,headsep=0cm,footskip=0cm}
\pagestyle{empty}
\setlength\parskip{\medskipamount}
\setlength\parindent{0pt}
\usepackage{setspace}
\onehalfspacing
\usepackage[authoryear]{natbib}

\makeatletter
\usepackage{url}

\usepackage{babel}
\makeatother

\makeatother

\begin{document}
\begin{center}\textbf{\LARGE Microeconomic Analysis with R}\end{center}{\LARGE \par}

\begin{center}Arne Henningsen\\
Department of Agricultural Economics\\
University of Kiel, Germany\end{center}

Since its first public release in 1993, the ``R language and environment
for statistical computing'' \citep{r-project} has been used more
and more for statistical analyses. 
However, in the first years it was not much used by economists and
 econometricians, but this situation has been changing in recent years. 
One cornerstone was the article ``R: Yet another econometric programming
environment'' by \citet{cribari99}.
Three years later \citet{racine02} published the article ``Using R to teach
econometrics''. 
And \citet{arai02} has a ``A brief guide to R for beginners in econometrics''
in the web that has been updated and improved several times within the past 
one and a half years.

Over the last years the number of R packages that are useful for economists
also increased. 
One of these packages is called ``systemfit'' \citep{systemfit} and provides
functions to estimate systems of linear or non-linear equations. 
Many economic models consist of several equations, which should be estimated
simultaneously, because the disturbance terms are likely contemporaneously
correlated or the economic theory requires cross-equation restrictions. 
This is especially the case in microeconomic modeling.

We extended the ``systemfit'' package to make it suitable for microeconomic
modeling (e.g. incorporating cross-equation restrictions).
Subsequently, we used it for several microeconomic demand and production
analyses. 
The demand analyses were done with the ``Almost Ideal Demand System'' (AIDS)
\citep{deaton80a} and the production analyses with the ``symmetric normalized
quadratic'' profit function. 
On the useR! conference we want to demonstrate this on a poster and on
a laptop computer. 
Furthermore a first release of a new R package will be presented that 
contains functions for microeconomic modeling (e.g. a function that carries 
out a full demand analysis with the AIDS with only a single command). 
Applied economists interested in microeconomic modeling will be invited to
contribute to this package by providing functions for other functional forms.

\bibliographystyle{apalike}
%\bibliography{/home/suapm095/Documents/Literatur/arne}
\bibliography{micEcon}

\end{document}
